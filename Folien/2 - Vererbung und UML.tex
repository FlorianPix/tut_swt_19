\documentclass[10pt]{beamer}

%theme
\usetheme{metropolis}
% packages
\usepackage{color}
\usepackage{listings}
\usepackage[main=ngerman, USenglish]{babel}
\usepackage[utf8]{inputenc}
\usepackage{multicol}
\usepackage{csquotes}
\usepackage{hyperref}

% color definitions
\definecolor{mygreen}{rgb}{0,0.6,0}
\definecolor{mygray}{rgb}{0.5,0.5,0.5}
\definecolor{mymauve}{rgb}{0.58,0,0.82}
\definecolor{paleorange}{HTML}{CC7832}

\lstset{
    backgroundcolor=\color{white},
    % choose the background color;
    % you must add \usepackage{color} or \usepackage{xcolor}
    basicstyle=\footnotesize\ttfamily,
    % the size of the fonts that are used for the code
    breakatwhitespace=false,
    % sets if automatic breaks should only happen at whitespace
    breaklines=true,                 % sets automatic line breaking
    captionpos=b,                    % sets the caption-position to bottom
    commentstyle=\color{mygreen},    % comment style
    % deletekeywords={...},
    % if you want to delete keywords from the given language
    extendedchars=true,
    % lets you use non-ASCII characters;
    % for 8-bits encodings only, does not work with UTF-8
    frame=single,                    % adds a frame around the code
    keepspaces=true,
    % keeps spaces in text,
    % useful for keeping indentation of code
    % (possibly needs columns=flexible)
    keywordstyle=\color{blue},       % keyword style
    % morekeywords={*,...},
    % if you want to add more keywords to the set
    numbers=left,
    % where to put the line-numbers; possible values are (none, left, right)
    numbersep=5pt,
    % how far the line-numbers are from the code
    numberstyle=\tiny\color{mygray},
    % the style that is used for the line-numbers
    rulecolor=\color{black},
    % if not set, the frame-color may be changed on line-breaks
    % within not-black text (e.g. comments (green here))
    stepnumber=1,
    % the step between two line-numbers.
    % If it's 1, each line will be numbered
    stringstyle=\color{mymauve},     % string literal style
    tabsize=4,                       % sets default tabsize to 4 spaces
    % show the filename of files included with \lstinputlisting;
    % also try caption instead of title
    language = Java,
	showspaces = false,
	showtabs = false,
	showstringspaces = false,
	escapechar = ,
        morecomment=[s][\textcolor{paleorange}]{@}{\ },
}

\def\ContinueLineNumber{\lstset{firstnumber=last}}
\def\StartLineAt#1{\lstset{firstnumber=#1}}
\let\numberLineAt\StartLineAt

\newcommand{\codeline}[1]{
        \alert{\texttt{#1}}
}

% Authors of the slides
\author{Florian Pix}

% Name of the Course
\institute{SWT Übung SoSe19}

% Presentation title
\title{Vererbung und UML}
\date{\today}

% Fancy Logo
\titlegraphic{\hfill\includegraphics[height=3cm]{Github_le_lemon}}

\begin{document}

\begin{frame}{Vererbung und UML}
    \titlepage
\end{frame}

\begin{frame}{Gliederung}
    \setbeamertemplate{section in toc}[sections numbered]
    \tableofcontents
\end{frame}

\begin{frame}[fragile]{Wiederholung}
    \section{Wiederholung}
\end{frame}

\begin{frame}[fragile]{Wiederholung}
    \begin{lstlisting}[basicstyle=\ttfamily\scriptsize,gobble=8]
         import java.util.Random;
         
         class Cell{
             private int status;
         
             Cell(){
                 Random rand = new Random();
                 this.status = rand.nextInt(2);
             }
         
             Cell(int status){  
                 this.status = status; 
             }
         
             public void setStatus(int status){  
                 this.status = status;
             }
         
             public int getStatus(){
                 return this.status;
             }
         }
    \end{lstlisting}
\end{frame}

\begin{frame}[fragile]{Objektattribut}
    \begin{lstlisting}[basicstyle=\ttfamily\scriptsize,gobble=8]
          private int status;
    \end{lstlisting}
\end{frame}

\begin{frame}[fragile]{Konstruktoren}
    \begin{lstlisting}[basicstyle=\ttfamily\scriptsize,gobble=8]
         Cell(){
             Random rand = new Random();
             this.status = rand.nextInt(2);
         }
     
         Cell(int status){  
             this.status = status; 
         }
    \end{lstlisting}
\end{frame}

\begin{frame}[fragile]{Getter}
    \begin{lstlisting}[basicstyle=\ttfamily\scriptsize,gobble=8]
          public int getStatus(){
              return this.status;
          }
    \end{lstlisting}
\end{frame}

\begin{frame}[fragile]{Setter}
    \begin{lstlisting}[basicstyle=\ttfamily\scriptsize,gobble=8]
          public void setStatus(int status){  
              this.status = status;
          }
    \end{lstlisting}
\end{frame}

\begin{frame}[fragile]{Konsole}
    \section{Konsole}
\end{frame}

\begin{frame}[fragile]{Konsole}
     \textcolor{mygreen}{\textbf{javac -d ./HelloLibrary *.java}}\newline

     javac\qquad\qquad\qquad Java Compiler\newline
    -d ./HelloLibrary\qquad Zielordner\newline
    *.java\qquad\qquad\qquad\quad zu kompilierende Dateien\newline
\end{frame}

\begin{frame}[fragile]{Konsole}
    Führe HelloLibrary.class aus
    
   \textcolor{mygreen}{\textbf{java HelloLibrary}}
\end{frame}

\begin{frame}[fragile]{Konsole}
    Lege jar an, mit default manifest.mf und \newline
    allen .class Datein im selben Ordner\newline
    \textcolor{mygreen}{\textbf{ jar cvf HelloLibrary.jar *.class}}\newline

    Wir müssen noch\newline
    \textcolor{mygreen}{\textbf{"Main-Class: HelloLibrary"}}\newline
    zum manifest hinzufügen.\newline
    \textcolor{mymauve}{\textbf{\href{https://docs.oracle.com/javase/tutorial/deployment/jar/appman.html}{Siehe Java Doc}\\}}

    \textcolor{mygreen}{\textbf{jar cfm HelloLibrary.jar manifest-addition.txt *.class}}\newline
    \textcolor{mymauve}{\textbf{\href{https://docs.oracle.com/javase/tutorial/deployment/jar/modman.html}{Siehe Java Doc}\\}}
\end{frame}

\begin{frame}[fragile]{UML und Aufgaben}
    \section{UML und Aufgaben}
\end{frame}

\end{document}