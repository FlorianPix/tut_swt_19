\documentclass[10pt]{beamer}

%theme
\usetheme{metropolis}
% packages
\usepackage{color}
\usepackage{tabularx}
\usepackage{listings}
\usepackage[main=ngerman, USenglish]{babel}
\usepackage[utf8]{inputenc}
\usepackage{multicol}
\usepackage{csquotes}
\usepackage{hyperref}

% color definitions
\definecolor{mygreen}{rgb}{0,0.6,0}
\definecolor{mygray}{rgb}{0.5,0.5,0.5}
\definecolor{mymauve}{rgb}{0.58,0,0.82}
\definecolor{paleorange}{HTML}{CC7832}

\lstset{
    backgroundcolor=\color{white},
    % choose the background color;
    % you must add \usepackage{color} or \usepackage{xcolor}
    basicstyle=\footnotesize\ttfamily,
    % the size of the fonts that are used for the code
    breakatwhitespace=false,
    % sets if automatic breaks should only happen at whitespace
    breaklines=true,                 % sets automatic line breaking
    captionpos=b,                    % sets the caption-position to bottom
    commentstyle=\color{mygreen},    % comment style
    % deletekeywords={...},
    % if you want to delete keywords from the given language
    extendedchars=true,
    % lets you use non-ASCII characters;
    % for 8-bits encodings only, does not work with UTF-8
    frame=single,                    % adds a frame around the code
    keepspaces=true,
    % keeps spaces in text,
    % useful for keeping indentation of code
    % (possibly needs columns=flexible)
    keywordstyle=\color{blue},       % keyword style
    % morekeywords={*,...},
    % if you want to add more keywords to the set
    numbers=left,
    % where to put the line-numbers; possible values are (none, left, right)
    numbersep=5pt,
    % how far the line-numbers are from the code
    numberstyle=\tiny\color{mygray},
    % the style that is used for the line-numbers
    rulecolor=\color{black},
    % if not set, the frame-color may be changed on line-breaks
    % within not-black text (e.g. comments (green here))
    stepnumber=1,
    % the step between two line-numbers.
    % If it's 1, each line will be numbered
    stringstyle=\color{mymauve},     % string literal style
    tabsize=4,                       % sets default tabsize to 4 spaces
    % show the filename of files included with \lstinputlisting;
    % also try caption instead of title
    language = Java,
	showspaces = false,
	showtabs = false,
	showstringspaces = false,
	escapechar = ,
        morecomment=[s][\textcolor{paleorange}]{@}{\ },
}

\def\ContinueLineNumber{\lstset{firstnumber=last}}
\def\StartLineAt#1{\lstset{firstnumber=#1}}
\let\numberLineAt\StartLineAt

\newcommand{\codeline}[1]{
        \alert{\texttt{#1}}
}

% Authors of the slides
\author{Florian Pix}

% Name of the Course
\institute{SWT Übung SoSe19}

% Presentation title
\title{Generics und Collections}
\date{25.04.2019}

% Fancy Logo
\titlegraphic{\hfill\includegraphics[height=3cm]{dreamfog}}

\begin{document}

\begin{frame}{Generics und Collections}
    \titlepage
\end{frame}

\begin{frame}{Gliederung}
    \setbeamertemplate{section in toc}[sections numbered]
    \tableofcontents
\end{frame}

\begin{frame}[fragile]{Generics}
    \section{Generics}
\end{frame}

\begin{frame}[fragile]{ohne Generics}
\begin{lstlisting}[basicstyle=\ttfamily\scriptsize,gobble=8]
		public class Box {
				private String contentS;
				private Integer contentI;
				private Cat contentC;

				public Box(String content){
						this.contentS = content;
				}

				public Box(Integer content){
						this.contentI = content;
				}

				public Box(Cat content){
						this.contentC = content;
				}
		}
\end{lstlisting}
\end{frame}

\begin{frame}[fragile]{mit Generics}
Parametrisierte Klasse:

\begin{lstlisting}[basicstyle=\ttfamily\scriptsize,gobble=8]
		public class Box<T> {
				private T content;

				public Box(T content){
						this.content = content;
				}
		}
\end{lstlisting}

Das T steht hier für einen beliebigen Typen.
\end{frame}

\begin{frame}[fragile]{Interfaces}
    \section{Interfaces}
\end{frame}

\begin{frame}[fragile]{Interfaces}
Ein  \textcolor{mygreen}{\textbf{Interface}} ist eine Schnittstelle, über die\\ 
einer Klasse bestimmte Funktionen \\
zur Verfügung gestellt werden.

Um diese Funktionen nutzen zu können, \\
 \textcolor{mymauve}{\textbf{müssen}} sie aber erst von der Klasse  \textcolor{mymauve}{\textbf{implementiert werden}}. \\
\end{frame}

\begin{frame}[fragile]{Interfaces}
Das  \textcolor{mygreen}{\textbf{Interface}} gibt nur den Rahmen bzw. \\
die \textcolor{mygray}{\textbf{Methodendeklarationen}} vor. \\
Es ist also nur eine \textcolor{mygreen}{\textbf{Blaupause}}, \\
und enthält deshalb ausschließlich \textcolor{mymauve}{\textbf{Konstanten}} und \\ 
\textcolor{mymauve}{\textbf{abstrakte Methoden}}. \\
\end{frame}

\begin{frame}[fragile]{Collections}
    \section{Collections}
\end{frame}

\begin{frame}[fragile]{Collections}
Java Collections sind Interfaces für Datenstrukturen.\\
Es gibt Listen, Sets, Maps und Queues.\\
Alle haben Vor- und Nachteile.\\
\textcolor{mymauve}{\textbf{\href{https://github.com/RatedARRR/TUT-JAVA-2018/blob/master/Slides/Javakurs 6-13.pdf}{Siehe alter Kurs}}}\\
Zu jedem sind zudem verschieden Implementierungen vorhanden.\\
\textcolor{mymauve}{\textbf{\href{https://docs.oracle.com/javase/8/docs/api/java/util/package-summary.html}{Siehe java.util}}}
\end{frame}

\end{document}