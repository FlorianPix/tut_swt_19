\documentclass[10pt]{beamer}

%theme
\usetheme{metropolis}
% packages
\usepackage{color}
\usepackage{tabularx}
\usepackage{listings}
\usepackage[main=ngerman, USenglish]{babel}
\usepackage[utf8]{inputenc}
\usepackage{multicol}
\usepackage{csquotes}
\usepackage{hyperref}
\usepackage{transparent}

% color definitions
\definecolor{mygreen}{rgb}{0,0.6,0}
\definecolor{mygray}{rgb}{0.5,0.5,0.5}
\definecolor{mymauve}{rgb}{0.58,0,0.82}
\definecolor{paleorange}{HTML}{CC7832}

\lstset{
    backgroundcolor=\color{white},
    % choose the background color;
    % you must add \usepackage{color} or \usepackage{xcolor}
    basicstyle=\footnotesize\ttfamily,
    % the size of the fonts that are used for the code
    breakatwhitespace=false,
    % sets if automatic breaks should only happen at whitespace
    breaklines=true,                 % sets automatic line breaking
    captionpos=b,                    % sets the caption-position to bottom
    commentstyle=\color{mygreen},    % comment style
    % deletekeywords={...},
    % if you want to delete keywords from the given language
    extendedchars=true,
    % lets you use non-ASCII characters;
    % for 8-bits encodings only, does not work with UTF-8
    frame=single,                    % adds a frame around the code
    keepspaces=true,
    % keeps spaces in text,
    % useful for keeping indentation of code
    % (possibly needs columns=flexible)
    keywordstyle=\color{blue},       % keyword style
    % morekeywords={*,...},
    % if you want to add more keywords to the set
    numbers=left,
    % where to put the line-numbers; possible values are (none, left, right)
    numbersep=5pt,
    % how far the line-numbers are from the code
    numberstyle=\tiny\color{mygray},
    % the style that is used for the line-numbers
    rulecolor=\color{black},
    % if not set, the frame-color may be changed on line-breaks
    % within not-black text (e.g. comments (green here))
    stepnumber=1,
    % the step between two line-numbers.
    % If it's 1, each line will be numbered
    stringstyle=\color{mymauve},     % string literal style
    tabsize=4,                       % sets default tabsize to 4 spaces
    % show the filename of files included with \lstinputlisting;
    % also try caption instead of title
    language = Java,
	showspaces = false,
	showtabs = false,
	showstringspaces = false,
	escapechar = ,
        morecomment=[s][\textcolor{paleorange}]{@}{\ },
}

\def\ContinueLineNumber{\lstset{firstnumber=last}}
\def\StartLineAt#1{\lstset{firstnumber=#1}}
\let\numberLineAt\StartLineAt

\newcommand{\codeline}[1]{
        \alert{\texttt{#1}}
}

% Authors of the slides
\author{Florian Pix}

% Name of the Course
\institute{SWT Übung SoSe19}

% Presentation title
\title{Vergleich von Datenstrukturen}
\date{02.05.2019}

% Fancy Logo
\titlegraphic{\hfill\includegraphics[height=3cm]{greek}}

\begin{document}

\begin{frame}{Generics und Collections}
    \titlepage
\end{frame}

\begin{frame}{Gliederung}
    \setbeamertemplate{section in toc}[sections numbered]
    \tableofcontents
\end{frame}

\begin{frame}[fragile]{Übersicht}
    \section{Übersicht}
\end{frame}

\begin{frame}[fragile]{Übersicht}
\includegraphics[height=5.25cm]{Model}
\end{frame}

\begin{frame}[fragile]{List}
    \section{List}
ArrayList

LinkedList
\end{frame}

\begin{frame}[fragile]{ArrayList}
Im Prinzip ein Array mit ein paar Methoden.

\begin{itemize}
\item Zugriff auf Elemente per Index \includegraphics[height=0.25cm]{uparrow}
\item Hinzufügen und Entfernen von Elementen langsam \includegraphics[height=0.25cm]{downarrow}
\end{itemize}

\textcolor{mymauve}{\textbf{\href{https://docs.oracle.com/javase/8/docs/api/java/util/ArrayList.html}{javadoc}}}
\end{frame}

\begin{frame}[fragile]{LinkedList}
Intern kein Array sondern eine doppelt verkettete Liste.\\
(Jedes Element hat einen Zeiger auf seinen Vorgänger und Nachfolger.)

\begin{itemize}
\item Einfügen und Entfernen von Elementen schnell \includegraphics[height=0.25cm]{uparrow}
\item Zugriff auf einzelnes Element langsam \includegraphics[height=0.25cm]{downarrow}
\item mehr Speicherbedarf \includegraphics[height=0.25cm]{downarrow}
\end{itemize}
\end{frame}

\begin{frame}[fragile]{Set}
    \section{Set}
HashSet

SortedSet
\end{frame}

\begin{frame}[fragile]{Set}
Realisierung der mathematischen Menge.\\
D.h. es gibt in einem Set keine Dublikate\\
aber auch keinen Index.
\end{frame}

\begin{frame}[fragile]{HashSet}
Einfügen und Suchen erfolgt mit Hashs.\\
Das sind int IDs, die aus den Daten\\
eines Objekts gebildet werden können.

\textcolor{mymauve}{\textbf{\href{http://www.straub.as/java/basic/hashset.html}{Hilfe}}}

\textcolor{mymauve}{\textbf{\href{https://www.baeldung.com/java-hashcode}{siehe hashCode()}}}
\end{frame}

\begin{frame}[fragile]{Einfügen in HashSets}
Mit dem \textcolor{mygreen}{HashCode} des einzufügenden Elementes sucht HashSet zuerst die Schublade mit dieser Nummer.
\end{frame}

\begin{frame}[fragile]{Einfügen in HashSets}
\texttransparent{0.4}{Mit dem \textcolor{mygreen}{HashCode} des einzufügenden Elementes sucht HashSet zuerst die Schublade mit dieser Nummer.}

Wird Sie nicht gefunden, wird das Element als neu eingestuft\\ 
und in einer neuen Schublade mit dieser Nummer abgelegt. 

Falls doch werden alle Objekte in dieser Schublade mit dem neuen Element verglichen.
 
Dazu verwendet \textcolor{mygreen}{HashSet} die Methode \textcolor{mygreen}{equals()}. 
\end{frame}

\begin{frame}[fragile]{Einfügen in HashSets}
\texttransparent{0.4}{Mit dem \textcolor{mygreen}{HashCode} des einzufügenden Elementes sucht HashSet zuerst die Schublade mit dieser Nummer.}

\texttransparent{0.4}{Wird Sie nicht gefunden, wird das Element als neu eingestuft\\ 
und in einer neuen Schublade mit dieser Nummer abgelegt.}

\texttransparent{0.4}{Falls doch werden alle Objekte in dieser Schublade mit dem neuen Element verglichen.}
 
\texttransparent{0.4}{Dazu verwendet \textcolor{mygreen}{HashSet} die Methode \textcolor{mygreen}{equals()}.}

Wird mit dieser Methode kein Objekt gefunden,\\ 
wird das Objekt in dieser Schublade gespeichert,\\ 
andernfalls wird es nicht aufgenommen. 
\end{frame}

\begin{frame}[fragile]{HashSet}
Daraus folgt dass wenn wir \textcolor{mygreen}{equals()} selbst implementieren, man auch die \textcolor{mygreen}{hashCode()} neu schreiben sollte.\\ 
Damit \textcolor{mygreen}{HashSets} effizient funktionieren.
\end{frame}

\begin{frame}[fragile]{HashSet}
\texttransparent{0.4}{Daraus folgt dass wenn wir \textcolor{mygreen}{equals()} selbst implementieren, man auch die \textcolor{mygreen}{hashCode()} neu schreiben sollte.\\ 
Damit \textcolor{mygreen}{HashSets} effizient funktionieren.}

Elemente, die mit \textcolor{mygreen}{equals} gleich sind, \\ 
müssen den gleichen \textcolor{mygreen}{HashCode} erzeugen. \\
Ansonsten würden sie in verschiedenen Schubladen gesteckt werden.
\end{frame}

\begin{frame}[fragile]{Sorted Set}
Elemente werden sortiert eingefügt nach ihrer Ordnung.\\
D.h. sie müssen Comparable implementieren.
\end{frame}

\begin{frame}[fragile]{Map}
    \section{Map}
\end{frame}

\begin{frame}[fragile]{Map}
Besteht aus Schlüssel-Wert-Paaren.\\
Werte können doppelt auftauchen,\\
aber Schlüssel müssen einzigartig sein.

\begin{itemize}
\item Werte könenn mit ihrem Schlüssel schnell gefunden werden.\includegraphics[height=0.25cm]{uparrow}
\end{itemize}
\end{frame}

\begin{frame}[fragile]{Vergleich}
\section{Vergleich}
\end{frame}

\begin{frame}[fragile]{Vergleich}
\includegraphics[height=5cm]{comp}
\textcolor{mymauve}{\textbf{\href{https://i0.wp.com/www.lavivienpost.com/wp-content/uploads/2018/04/Java-collections-cheat-sheet2.jpg?resize=768 2C500&ssl=1}{Source}}}
\end{frame}

\end{document}