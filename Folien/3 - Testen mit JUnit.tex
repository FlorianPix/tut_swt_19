\documentclass[10pt]{beamer}

%theme
\usetheme{metropolis}
% packages
\usepackage{color}
\usepackage{tabularx}
\usepackage{listings}
\usepackage[main=ngerman, USenglish]{babel}
\usepackage[utf8]{inputenc}
\usepackage{multicol}
\usepackage{csquotes}
\usepackage{hyperref}

% color definitions
\definecolor{mygreen}{rgb}{0,0.6,0}
\definecolor{mygray}{rgb}{0.5,0.5,0.5}
\definecolor{mymauve}{rgb}{0.58,0,0.82}
\definecolor{paleorange}{HTML}{CC7832}

\lstset{
    backgroundcolor=\color{white},
    % choose the background color;
    % you must add \usepackage{color} or \usepackage{xcolor}
    basicstyle=\footnotesize\ttfamily,
    % the size of the fonts that are used for the code
    breakatwhitespace=false,
    % sets if automatic breaks should only happen at whitespace
    breaklines=true,                 % sets automatic line breaking
    captionpos=b,                    % sets the caption-position to bottom
    commentstyle=\color{mygreen},    % comment style
    % deletekeywords={...},
    % if you want to delete keywords from the given language
    extendedchars=true,
    % lets you use non-ASCII characters;
    % for 8-bits encodings only, does not work with UTF-8
    frame=single,                    % adds a frame around the code
    keepspaces=true,
    % keeps spaces in text,
    % useful for keeping indentation of code
    % (possibly needs columns=flexible)
    keywordstyle=\color{blue},       % keyword style
    % morekeywords={*,...},
    % if you want to add more keywords to the set
    numbers=left,
    % where to put the line-numbers; possible values are (none, left, right)
    numbersep=5pt,
    % how far the line-numbers are from the code
    numberstyle=\tiny\color{mygray},
    % the style that is used for the line-numbers
    rulecolor=\color{black},
    % if not set, the frame-color may be changed on line-breaks
    % within not-black text (e.g. comments (green here))
    stepnumber=1,
    % the step between two line-numbers.
    % If it's 1, each line will be numbered
    stringstyle=\color{mymauve},     % string literal style
    tabsize=4,                       % sets default tabsize to 4 spaces
    % show the filename of files included with \lstinputlisting;
    % also try caption instead of title
    language = Java,
	showspaces = false,
	showtabs = false,
	showstringspaces = false,
	escapechar = ,
        morecomment=[s][\textcolor{paleorange}]{@}{\ },
}

\def\ContinueLineNumber{\lstset{firstnumber=last}}
\def\StartLineAt#1{\lstset{firstnumber=#1}}
\let\numberLineAt\StartLineAt

\newcommand{\codeline}[1]{
        \alert{\texttt{#1}}
}

% Authors of the slides
\author{Florian Pix}

% Name of the Course
\institute{SWT Übung SoSe19}

% Presentation title
\title{Testen mit JUnit}
\date{18.04.2019}

% Fancy Logo
\titlegraphic{\hfill\includegraphics[height=3cm]{blue}}

\begin{document}

\begin{frame}{Testen mit JUnit}
    \titlepage
\end{frame}

\begin{frame}{Hilfe}
    \textcolor{mymauve}{\textbf{\href{https://st.inf.tu-dresden.de/files/teaching/ss19/st/Arbeiten mit JUnit 3.8.pdf}{Anleitung der Fakultät zum Einstieg in JUnit}}}
\end{frame}

\begin{frame}{Gliederung}
    \setbeamertemplate{section in toc}[sections numbered]
    \tableofcontents
\end{frame}

\begin{frame}[fragile]{Testfalltabellen}
    \section{Testfalltabellen}
\end{frame}

\begin{frame}[fragile]{Testfalltabellen}
\begin{description}
\item[Bestimme die Zustände:]\hfill \\
in denen sich ein Objekt\\ 
nach einer Anweisung befinden muss
\item[Bestimme die Extremwerte:]\hfill \\
der Parameter der zu testenden Methode \\
z.B. 0 oder null
\item[Bestimme die Randwerte:]\hfill \\
der Parameter der zu testenden Methode \\
z.B. 1.1. und 31.12.
\end{description}
\end{frame}

\begin{frame}[fragile]{Testfalltabellen}
Testfalltabelle für die Methode grow(int move) aus einem Snakespiel

\begin{tabularx}{11cm}{|X|X|X|X|X|}
\hline
Testfall & Erwarteter Status & Klasse & Parameter & Erwartete Ausgabe\\
\hline
S1 & Exception & Snake & -1 & Eine Schlange kann nicht schrumpfen.\\
\hline
S2 & ok & Snake & 1 & 2\\
\hline
S3 & ok & Snake & 5 & 6\\
\hline
\end{tabularx}
\end{frame}

\begin{frame}[fragile]{Fixtures}
    \section{Fixtures}
\end{frame}

\begin{frame}[fragile]{Fixtures}
    Eine Fixierung ist ein fixer Status von Objekten der als Grundlagen des Durchlaufen von Tests genutzt wird. \\
    Der Sinn einer solchen Fixierung ist es sicher zu stellen dass es eine bekannte Umgebung gibt in der Resultate reproduzierbar sind. \\
    Beispiel:
    \begin{description}
    \item[Vorbereitung von Input Daten:]\hfill \\
        Erstellung von Mock-Objekten\\
        Laden einer Datenbank mit bekannten Datensatz
    \end{description}
\end{frame}

\begin{frame}[fragile]{Fixtures}
    Dazu muss ebenfalls sichergestellt werden dass bei einem Test die Umgebung nicht so verändert wurde dass andere Tests beeinflusst werden könnten.
\end{frame}

\begin{frame}[fragile]{Fixtures}
    JUnit 3.8.4\\
    \textcolor{mygreen}{\textbf{setUp()}} \newline
    \textcolor{mygreen}{\textbf{tearDown()}}
\end{frame}

\begin{frame}[fragile]{Testrunner}
    \section{Testrunner}
\end{frame}

\begin{frame}[fragile]{Testrunner}
    Testeinstieg/Main-Methode\\
    Beispiele:\\
    \textcolor{mygreen}{\textbf{junit.textui.TestRunner}} \newline
    \textcolor{mygreen}{\textbf{junit.swingui.TestRunner}}
\end{frame}

\begin{frame}[fragile]{Testsuite}
    \section{Testsuite}
\end{frame}

\begin{frame}[fragile]{Testsuite}
Wenn wir viele Tests auf einmal ausführen wollen können wir diese in einer Testsuite zusammenfassen.\\
\begin{lstlisting}[basicstyle=\ttfamily\scriptsize,gobble=8]
        TestSuite suite= new TestSuite();
        suite.addTest(new FirstTest("testEquals"));
        suite.addTest(new FirstTest("testSimpleAdd"));
        TestResult result= suite.run();
\end{lstlisting}
Oder:
\begin{lstlisting}[basicstyle=\ttfamily\scriptsize,gobble=8]
        TestSuite suite= new TestSuite(FirstTest.class); 
        TestResult result= suite.run();
\end{lstlisting}
\end{frame}


\begin{frame}[fragile]{JUnit 3 und 4}
    \section{JUnit 3 und 4}
\end{frame}

\begin{frame}[fragile]{JUnit 3 und 4}
    \hfill\includegraphics[scale=0.45]{comparison}
    \textcolor{mymauve}{\textbf{\href{http://asjava.com/junit/junit-3-vs-junit-4-comparisonf}{Quelle: "asjava.com"}}}\\
    \textcolor{mymauve}{\textbf{\href{https://github.com/RatedARRR/TUT-JAVA-2018/blob/master/Slides/Javakurs 11-13.pdf}{Hilfe zu JUnit4}}}
\end{frame}

\end{document}